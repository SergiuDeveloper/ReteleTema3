- single point of error : the TCP server
- Encryption : SHA256
- singletons

Architecture : 
Server sided:
- one TCP Server that handles the outside requests and informs the other nodes of the tasks they should perform; echo back to requester
	each user is identified by his IP and his MAC address, as there could be more clients connected with the same IP
	why TCP? we gotta make sure the info gets there correctly
- one local AF_UNIX server that handles requests from the command line;
	why making one more server? because the TCP server process is being detached from the console, so that the user can perform other operations
	why AF_UNIX? In order to skip the network layers that TCP implies, as AF_UNIX doesn't need a binded port or IP address, because it binds a certain physical disk location
- local AF_UNIX client

Client sided:
- one TCP Client that can send requests if he's admin or execute requests that were redirected by the TCP Server towards them
- one local AF_UNIX server for the same purpose
- local AF_UNIX client

+- Remote desktop connection
Architecture:
- the connection requests are being handled by the TCP Server mentioned above
- after a connection is made, an UDP server handles the transmission of visual data(the user's desktop)
- another TCP handles, dually, the commands that are to be executed on the screen

All the server mentioned are configured after the TCP(/UDP/Local) concurrent Server-Client pattern